%%%%%%%%%%%%%%%%%%%%%%%%%%%%%%%%%%%%%%%%%%%%%%%%%%%%%%%%%%%%%%%%%%%%%%%%%%%%%%%%
% TUM-Vorlage: Präsentation
%%%%%%%%%%%%%%%%%%%%%%%%%%%%%%%%%%%%%%%%%%%%%%%%%%%%%%%%%%%%%%%%%%%%%%%%%%%%%%%%
%
% Rechteinhaber:
%     Technische Universität München
%     https://www.tum.de
% 
% Gestaltung:
%     ediundsepp Gestaltungsgesellschaft, München
%     http://www.ediundsepp.de
% 
% Technische Umsetzung:
%     eWorks GmbH, Frankfurt am Main
%     http://www.eworks.de
%
%%%%%%%%%%%%%%%%%%%%%%%%%%%%%%%%%%%%%%%%%%%%%%%%%%%%%%%%%%%%%%%%%%%%%%%%%%%%%%%%


%%%%%%%%%%%%%%%%%%%%%%%%%%%%%%%%%%%%%%%%%%%%%%%%%%%%%%%%%%%%%%%%%%%%%%%%%%%%%%%%
% Zur Wahl des Seitenverhältnisses bitte einen der beiden folgenden Befehle
% auskommentieren und den ausführen lassen:
\input{../template/res/Praeambel16zu9.tex} % Seitenverhältnis 16:9

%%%%%%%%%%%%%%%%%%%%%%%%%%%%%%%%%%%%%%%%%%%%%%%%%%%%%%%%%%%%%%%%%%%%%%%%%%%%%%%%


%%%%%%%%%%%%%%%%%%%%%%%%%%%%%%%%%%%%%%%%%%%%%%%%%%%%%%%%%%%%%%%%%%%%%%%%%%%%%%%%
\input{../_Einstellungen.tex}                    % !!! DATEI ANPASSEN !!!
%%%%%%%%%%%%%%%%%%%%%%%%%%%%%%%%%%%%%%%%%%%%%%%%%%%%%%%%%%%%%%%%%%%%%%%%%%%%%%%%


% \renewcommand{\PersonTitel}{Dr. rer. nat.}
\newcommand{\Datum}{\today}

\renewcommand{\PraesentationFusszeileZusatz}{| Tutorium Einführung in die Rechnerarchitektur WS 2018/2019}

\title{Tutorübung 1}
\author{\PersonVorname{} \PersonNachname}
\institute[]{\UniversitaetName \\ \FakultaetName}
\date[\Datum]{15. Oktober 2018}


%%%%%%%%%%%%%%%%%%%%%%%%%%%%%%%%%%%%%%%%%%%%%%%%%%%%%%%%%%%%%%%%%%%%%%%%%%%%%%%%
\input{../template/res/Anfang.tex} % !!! NICHT ENTFERNEN !!!
%%%%%%%%%%%%%%%%%%%%%%%%%%%%%%%%%%%%%%%%%%%%%%%%%%%%%%%%%%%%%%%%%%%%%%%%%%%%%%%%
\begin{document}
\setlength{\baselineskip}{\PraesentationAbstandAbsatz}
\setlength{\parskip}{\baselineskip}

%%%%%%%%%%%%%%%%%%%%%%%%%%%%%%%%%%%%%%%%%%%%%%%%%%%%%%%%%%%%%%%%%%%%%%%%%%%%%%%%
% FOLIENSTIL: Standard
% !!!ÄNDERUNG HIER:!!!
\PraesentationMasterStandard

\PraesentationTitelseite % Fügt die Startseite ein

\begin{frame}
    \frametitle{Aufgabe 1}
    \vspace{0.5cm}

    \begin{enumerate}[(a)]
        \item<1-> Wie kann man überprüfen, ob eine Zahl gerade oder ungerade ist?
        \item<2->[] Lösung
        \item<3-> Wie kann man eine bestimmte Bitposition überprüfen 
            oder auf 0 bzw. 1 setzen (”zwingen“)? 
            Warum geht das Erzwingen auf 1 nicht universell mit einer Addition?
        \item<4->[] Lösung 
    \end{enumerate}
\end{frame}

\begin{frame}
    \frametitle{Aufgabe 2}
    \vspace{0.5cm}

    \begin{enumerate}[(a)]
        \item<1-> Wodurch kann der 80386 Befehl NEG EAX ersetzt werden?
        \item<2->[] Lösung
        \item<3-> Wie addiert man eine vorzeichenlose Zahl im Register AL korrekt auf eine Zahl im Register EBX?
        \item<4->[] Lösung
        \item<5-> Man übersetze folgende Berechnung in ein 80386-Programm: \\
           \begin{center}
                \( 24 \cdot a + b + c + 1234 \)
            \end{center}
            Dabei sind \( a,b,c \) vorzeichenlose 32-Bit-Werte und liegen bereits in den 
            Registern EAX, EBX bzw. ECX vor. Das Ergebnis soll in EDX stehen.
        \item<6->[] Lösung
        \item<5-> Die Division des 80386 legt den Divisionsrest (Remainder)
            im Register EDX ab, führt also eine Modulo-Berechnung durch.
            Für welche Teiler könnte man die Modulo- Berechnung wesentlich schneller durchführen? 
            Welche logische Funktion würde sich anbieten?
        \item
    \end{enumerate}
\end{frame}

\begin{frame}[fragile]
    \frametitle{Aufgabe 3}
    \vspace{0.5cm}

    Anhand der Intel-Dokumentation sollen die zu den Opcodes für die Befehle 
    gehörenden Bytefolgen (als Hexwerte) zusammengebaut werden. 
    Dies entspricht einem ”Assemblieren“ von Hand von:

    ADD EAX, 0x12345678 \\
    ADD EAX, EBX \\
    MOV AX, 0x10 \\

\end{frame}

\begin{frame}
    \frametitle{Aufgabe 4: Festkommarechnung}
    \vspace{0.5cm}

    \begin{enumerate}[(a)]
        \item Was passiert, wenn man die in der Zentralübung erwähnte Formel zur Berechnung
        des Werts einer Binärzahl \( (a = \sum_{i=0}^{n} a_i \cdot 2^i ) \) auf negative Indices erweitert?
        \item Welchen Wertebereich kann man mit 8 binären Vorkommastellen (ohne Vorzeichen)
        und 8 binären Nachkommastellen (8.8) erreichen?
        \item 
        \item 
    \end{enumerate}
\end{frame}

%%%%%%%%%%%%%%%%%%%%%%%%%%%%%%%%%%%%%%%%%%%%%%%%%%%%%%%%%%%%%%%%%%%%%%%%%%%%%%%%
\end{document} % !!! NICHT ENTFERNEN !!!
%%%%%%%%%%%%%%%%%%%%%%%%%%%%%%%%%%%%%%%%%%%%%%%%%%%%%%%%%%%%%%%%%%%%%%%%%%%%%%%%
